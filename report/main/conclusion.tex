\chapter{Conclusion}
\paragraph{}
Statistics usually revolves around the behaviour of the mean. The central limit theorem is an example of this. Extreme value theory takes a very different approach, as it is the study of the behaviour of the maxima. The central limit theorem finds its extreme value counterpart in the Fisher-Tippett-Gnedenko and Von Mises' theorems. Together they make up the answer to the extremal problems : knowing what the possible limiting distributions are and knowing under which conditions there is convergence to those distributions.
\paragraph{}
Using the Black-Scholes with time independent coefficients is actually a rather good model for the simulation of stock prices, on the condition that the actual stock is not too volatile. As we have been able to see, should the actual stock display huge jumps for instance, the simulation would eventually be a poor approximation of the actual prices. Beyond the scope of the project, let us mention that stochastic jump processes are usually considered a better, more sophisticated model for stock prices. This is something we now set out to study more in-depth.\newline
A result of note bridging the gap between on the one hand extreme value theory and finance on the other hand is that it if we consider the stock prices above a high quantile (\textit{e.g.} $95$ \% or $97.5$ \%), the empirical distributions fit extreme values distributions.

\bigskip
\paragraph{}
From a more personal perspective, I must say that this project has been a wonderful experience. It was the opportunity to get back to R, to do more stochastic calculus and to get started in extreme value theory. Last but not least, following a course in class, with lecture notes that were originally conceived as such is not always easy. Working on one's own on reference books, the scope of which is often much broader than what one would need is another experience entirely however. This project has taught me precious lessons, and I will not forget them.