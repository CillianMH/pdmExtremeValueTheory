\chapter{Conclusion}
\paragraph{}
Statistics usually revolves around the behaviour of the mean. The central limit theorem is an example of this. Extreme value theory takes a very different approach, as it is the study of the behaviour of the maxima. The central limit theorem finds its extreme value counterpart in the Fisher-Tippett-Gnedenko and Von Mises' theorems. Together they make up the answer to the extremal problems : knowing what the possible limiting distributions are and knowing under which conditions there is convergence to those distributions.
\paragraph{}
Using the Black-Scholes with time independent coefficients is actually a rather good model for the simulation of stock prices \textbf{\textcolor{red}{on the whole}}, on the condition that the actual stock is not too volatile. As we have been able to see, should the actual stock display huge jumps for instance, the simulation would eventually be a poor approximation of the actual prices. Beyond the scope of the project, let us mention that stochastic jump processes are usually considered a better, more sophisticated model for stock prices. This is something we now set out to study more in-depth.
\paragraph{}
Finally, we have shown that if we consider the log-returns above a high quantile or the minus log-returns below a low quantile, they fit extreme value distributions. We expected those distributions to be Gumbel-type distributions, as it should be if the Black-Scholes model were valid in extreme context, for in that case the log-returns and the minus log-returns would follow Gaussian distributions. For each of the stocks, we have computed the maximum-likelihood of the shape parameter, $\hat{\gamma}_{MLE}$. Even accounting for the standard errors of those MLE estimates, the values $\gamma_{MLE}$ taken by $\hat{\gamma}_{MLE}$ are not consistent with Gumbel-type distributions. We can thus conclude that the Black-Scholes model is not an adequate model for  \textbf{\textcolor{red}{extreme values}} in finance.
\bigskip
\paragraph{}
From a more personal perspective, I must say that this project has been a wonderful experience. It was the opportunity to get back to R, to do more stochastic calculus and to get started in extreme value theory. Last but not least, following a course in class, with lecture notes that were originally conceived as such is not always easy. Working on one's own on reference books, the scope of which is often much broader than what one would need is another experience entirely, however. This project has taught me precious lessons, and I will not forget them.