\chapter{Looking into real-world data}
\addcontentsline{toc}{chapter}{Looking into real-world data}
\section{Five real-world stocks and their evolution over 15 years}
We have chosen to study five stocks listed on the Paris Stock Exchange : BNP Paribas, Carrefour, LVMH, Sanofi and Total stocks.\footnote{We have chosen companies positioned on different domains, otherwise, information from different stock might more easily be redundant.} The evolution of the stock prices has been studied over the past 15 years, on a weekly basis. We first draw the data itself, then the net returns and the gross log returns on the stocks\footnote{Both quantities are widely used in Finance.}
\begin{figure}[h!]
	\centering
	\begin{minipage}[b]{0.4\textwidth}
 	\centering
 	\includegraphics[scale = 0.4]{/Users/kimartin/Desktop/PDM_thesis_report/latex_template5_57/main/R_Files_3/WeeklyBNP.jpeg}
 	\caption{15 years of weekly BNP Stock Price Data}
 	\label{fig:BNPStock}
	\end{minipage}
	~
	\begin{minipage}[b]{0.4\textwidth}
  	\centering
  	\includegraphics[scale = 0.4]{/Users/kimartin/Desktop/PDM_thesis_report/latex_template5_57/main/R_Files_3/WeeklyCarrefour.jpeg}
  	\caption{15 years of weekly Carrefour Stock Price Data}
  	\label{fig:CarrefourStock}
	\end{minipage}
\end{figure}
\begin{figure}[h!]
	\centering
	\begin{minipage}[b]{0.4\textwidth}
   	\centering
   	\includegraphics[scale = 0.4]{/Users/kimartin/Desktop/PDM_thesis_report/latex_template5_57/main/R_Files_3/WeeklyLVMH.jpeg}
   	\caption{15 years of weekly LVMH Stock Price Data}
   	\label{fig:LVMHStock}
	\end{minipage}
	~
	\begin{minipage}[b]{0.4\textwidth}
    	\centering
    	\includegraphics[scale = 0.4]{/Users/kimartin/Desktop/PDM_thesis_report/latex_template5_57/main/R_Files_3/WeeklySanofi.jpeg}
    	\caption{15 years of weekly Sanofi Stock Price Data}
    	\label{fig:SanofiStock}
	\end{minipage}
	~
	\begin{minipage}[b]{0.4\textwidth}
     	\centering
     	\includegraphics[scale = 0.4]{/Users/kimartin/Desktop/PDM_thesis_report/latex_template5_57/main/R_Files_3/WeeklyTotal.jpeg}
     	\caption{15 years of weekly Total Stock Price Data}
     	\label{fig:TotalStock}
	\end{minipage}
\end{figure}
\newpage
\paragraph{}
Let $X_t$ be the price of a stock at time t, the gross return at time t + 1 is defined as the ratio $\frac{X_{t+1}}{X_t}$, the net return at time t + 1 is defined as the ratio $r_t = \frac{X_{t+1}-X_t}{X_t}$ and the log gross return at time t +1 is defined as the log of the gross return at time t + 1 i.e. $R_t = \log(\frac{X_{t+1}}{X_t})$. The latter two quantities are of particular interest in Finance. 
\newline
Let us observe that the relationship between $R_t$, $X_t$ and $X_{t+1}$ can be rewritten as $X_{t+1} = \exp(R_{t+1})*X_t$. An approximation would be to take $X_{t+1} = (1 + R_{t+1})*X_t$ by taking the expansion of the exponential, cut at order 1. Below are the plots of the quantities $\exp(R_t)$ and 1 + $R_t$ for the five stocks previously considered. As we can see from the value of the residuals, this is in practice a very good approximation !
\begin{figure}[h!]
	\centering
	\begin{minipage}[b]{0.4\textwidth}
		\centering
		\includegraphics[scale = 0.4]{/Users/kimartin/Desktop/PDM_thesis_report/latex_template5_57/main/R_Files_3/WeeklyBNP_1+Rt_eRt.jpeg}
		\caption{$\exp(R_t)$ and 1 + $R_t$ for BNP Stock Price Data, residual : 3.22E-15}
		\label{fig:BNPCompApproxStock}
	\end{minipage}
	~
	\begin{minipage}[b]{0.4\textwidth}
		\centering
		\includegraphics[scale = 0.4]{/Users/kimartin/Desktop/PDM_thesis_report/latex_template5_57/main/R_Files_3/WeeklyCarrefour_1+Rt_eRt.jpeg}
		\caption{$\exp(R_t)$ and 1 + $R_t$ for Carrefour Stock Price Data, residual : 1.44E-15}
		\label{fig:CarrefourCompApproxStock}
	\end{minipage}
\end{figure}
\begin{figure}[h!]
	\centering
	\begin{minipage}[b]{0.4\textwidth}
		\centering
		\includegraphics[scale = 0.4]{/Users/kimartin/Desktop/PDM_thesis_report/latex_template5_57/main/R_Files_3/WeeklyLVMH_1+Rt_eRt.jpeg}
		\caption{$\exp(R_t)$ and 1 + $R_t$ for LVMH Stock Price Data, residual : 4.75E-15}
		\label{fig:LVMHCompApproxStock}
	\end{minipage}
	~
	\begin{minipage}[b]{0.4\textwidth}
		\centering
		\includegraphics[scale = 0.4]{/Users/kimartin/Desktop/PDM_thesis_report/latex_template5_57/main/R_Files_3/WeeklySanofi_1+Rt_eRt.jpeg}
		\caption{$\exp(R_t)$ and 1 + $R_t$ for Sanofi Stock Price Data, residual : 1.55E-15}
		\label{fig:SanofiCompApproxStock}
	\end{minipage}
	~
	\begin{minipage}[b]{0.4\textwidth}
		\centering
		\includegraphics[scale = 0.4]{/Users/kimartin/Desktop/PDM_thesis_report/latex_template5_57/main/R_Files_3/WeeklyTotal_1+Rt_eRt.jpeg}
		\caption{$\exp(R_t)$ and 1 + $R_t$ for Total Stock Price Data, residual : 3.86E-15}
		\label{fig:TotalCompApproxStock}
	\end{minipage}
\end{figure}
\paragraph{}

\newpage
\section{A détour around Stochastic Calculus}
\subsection{The Black-Scholes Stochastic Differential Equation}
\paragraph{Presentation}
It is customary to model the evolution of stock prices by a stochastic process $(S_t)_{t \ge 0}$ satisfying the Black-Scholes stochastic differential equation, which rea\mathrm{d}s as follows : 
\begin{equation}
\mathrm{d}S_t = \mu S_t \mathrm{d}t + \sigma S_t \mathrm{d}B_t
\end{equation}
where $(B_t)_{t \ge 0}$ is a standard Brownian Motion with respect to a filtration $({\cal{F}}_t)_{t \ge 0}$. Let us observe that the stochastic differential equation above is the Black-Scholes SDE with time independent coefficients : both the drift $\mu \in \mathbb{R}$ and the volatility $\sigma > 0$ are constant.\footnote{In the next subsection, we will deal with the Black-Scholes SDE with time dependent coefficients.} An initial condition must be specified : $S_0 = s_0 > 0$.
\paragraph{Resolution - existence of a solution to the Black-Scholes SDE}
Let us consider the following generic stochastic differential equation : \\
$\left\{
\begin{array}{l}
\mathrm{d}X_t = f(X_t) \mathrm{d}t + g(X_t) \mathrm{d}B_t\\
X_0 = x_0
\end{array}
\right.$ \\ where $(B_t)_{t \ge 0}$ is a standard Brownian Motion with respect to a filtration $({\cal{F}}_t)_{t \ge 0}$, $x_0 \in \mathbb{R}$, $f,g : \mathbb{R} \longrightarrow \mathbb{R}$ are Lipschitz functions. Then, by a theorem from Stochastic Calculus\footnote{readers wanting to get ahold of an excellent course on Stochastic Calculus are advised to refer to Dr. Lév\^{e}que's course. It can be found at \url{http://ipg.epfl.ch/~leveque/}}, we know that there exists a unique process that is continuous and adapted to the filtration $({\cal{F}}_t)_{t \ge 0}$. \\
Here, f and g are respectively the functions $x \longrightarrow \mu x$ and $x \longrightarrow \sigma t$. These functions are Lipschitz functions, therefore we know the SDE admits a solution. And fortunately, in the case of the Black-Scholes equation, the solution can be made explicit\footnote{This is not always the case : solving SDEs in general is not that simple, and finding an explicit solution is not guaranteed in the general case, even though we know one exists.}).
\paragraph{Resolution - step 1}
Let us set $S_t = \phi_t Z_t$ where $\phi_t$ is the (deterministic) solution to the ordinary differential equation :
\\
$\left\{
\begin{array}{l}
\mathrm{d}\phi_t = \mu \phi_t \mathrm{d}t\\
\phi_0 = 1
\end{array}
\right.$ \\ Solving this ODE is elementary and yiel\mathrm{d}s the solution $\phi_t = \exp(\mu t)$. Now, let us differentiate $X_t$ under its form as a product  $S_t = \phi_t Z_t$. \\
\begin{equation}
\begin{alignat*}{2}
\mathrm{d}(S_t) &= \mathrm{d}(\phi_t Z_t)\\ 
&= \phi_t \mathrm{d}Z_t + Z_t \mathrm{d}\phi_t + \mathrm{d}<\phi,Z>_t\\ 
&= \phi_t \mathrm{d}Z_t + \mu \phi_t Z_t \mathrm{d}t\\
&= \phi_t \mathrm{d}Z_t + \mu S_t \mathrm{d}t\\
\end{alignat*}
\end{equation} where the infinitesimal quadratic covariation between $\phi$ and $Z$ is zero as $\phi$ has bounded variations. If we compare the last right-hand term of the series of equations just above to the original Black-Scholes SDE, we see that $\sigma S_t \mathrm{d}B_t = \phi_t \mathrm{d}Z_t$. Hence, $\mathrm{d}Z_t = \sigma Z_t \mathrm{d}B_t$. Here, we must be very careful as this differential equation does not integrate as it would in the settings of 'usual' calculus, in particular, integrating it to $\log(Z_t) - \log(Z_0) = \sigma (B_t - B_0)$ is \underline{totally wrong !}
\paragraph{Resolution - step 2}
Now, let us set $Y_t = \log(Z_t)$. Using Ito-Doeblin's formula, we get : \\
$d(\log(Z_t)) = \frac{1}{Z_t} \mathrm{d}Z_t + \frac{1}{2} (\frac{-1}{Z^2_{t}})\mathrm{d}<Z>_t$ ($\star$)\\
\underline{$\rightarrow$ How to compute $\mathrm{d}<Z>_t$ ?}
\begin{equation}
\begin{alignat*}{2}
S_t &= \phi_t Z_t\\ 
\implies Z_t &= \frac{S_t}{\phi_t}\\ 
\implies \mathrm{d}Z_t &= \frac{1}{\phi_t} \mathrm{d}S_t + S_t d(\frac{1}{\phi_t}) + \mathrm{d}<\frac{1}{\phi},S>_t\\
&= \frac{1}{\phi_t} \mathrm{d}S_t - \frac{S_t}{\phi_t^2} d(\phi_t) \\
&= \frac{1}{\phi_t} \mathrm{d}S_t - \frac{\mu S_t}{\phi_t} \mathrm{d}t \\
&= \frac{1}{\phi_t} (\mu S_t \mathrm{d}t + \sigma S_t \mathrm{d}B_t) - \frac{\mu S_t}{\phi_t} \mathrm{d}t \\
&= \frac{\sigma S_t}{\phi_t} \mathrm{d}B_t 
\end{alignat*}
\end{equation} where the infinitesimal quadratic covariation between $\frac{1}{\phi}$ and $Z$ is zero as $\frac{1}{\phi}$ has bounded variations, in the third equality from the top.\newline
Hence, using the Isometry formula, we have that :
\begin{alignat*}{2}
<Z>_t &= \int_0^t \! \frac{\sigma^2 S_s^2}{\phi_s^2} \, \mathrm{d}s \\ 
 &= \int_0^t \! \sigma^2 Z_s^2 \, \mathrm{d}s \\ 
\implies \mathrm{d}<Z>_t &= \sigma^2 Z_t^2 \mathrm{d}t
\end{alignat*}
\end{equation} \newline
Back to ($\star$), we now have :
\begin{equation}
\begin{alignat*}{2}
\mathrm{d}(\log(Z_t)) &= \frac{1}{Z_t} \mathrm{d}Z_t + \frac{1}{2} (\frac{-1}{Z^2_{t}}) \sigma^2 Z^2_{t} \mathrm{d}t \\
\iff \frac{1}{Z_t} \mathrm{d}Z_t &= d(\log(Z_t)) + \frac{\sigma^2}{2} \mathrm{d}t 
\end{alignat*}
\end{equation} 
\paragraph{Resolution - step 3}
Combining the previous equation with $\mathrm{d}Z_t = \sigma Z_t \mathrm{d}B_t$, we get : \newline
\begin{equation}
\begin{alignat*}{2}
\sigma \mathrm{d}B_t &= \mathrm{d}(log(Z_t)) + \frac{\sigma^2}{2} \mathrm{d}t \\
\implies \log(Z_t) -\log(Z_0) &= -\frac{\sigma^2}{2} (t - 0) + \sigma (B_t - B_0) \\
\implies \log(Z_t) -\log(\frac{S_0}{\phi_0}) &= -\frac{\sigma^2}{2} t + \sigma B_t \\
\implies \log(Z_t) &= \log(s_0) -\frac{\sigma^2}{2} t + \sigma B_t \\
\implies Z_t &= s_0 \exp(-\frac{\sigma^2}{2} t + \sigma B_t )
\end{alignat*}
\end{equation} \newline Finally, by remembering that $S_t = \phi_t Z_t = \exp(\mu t) Z_t$, we get :\newline
$\forall t \ge 0, S_t = s_0 \exp((\mu -\frac{\sigma^2}{2}) t + \sigma B_t)$
\newline The stochastic process $(S_t)_{t \ge 0}$, made explicit above, that is solution to the Black-Scholes equation is generally called Geometric Brownian Motion in the literature.
\subsection{Black-Scholes SDE with time-dependent coefficients }
\paragraph{Presentation} In the simple Black-Scholes SDE, the drift $\mu$ and the volatility $\sigma$ were time-independent constants. Let us now consider a more general version of the Black-Scholes SDE : \newline
$\left\{
\begin{array}{l}
\mathrm{d}S_t = \mu(t) S_t \mathrm{d}t + \sigma(t) S_t \mathrm{d}B_t\\
S_0 = s_0 > 0
\end{array}
\right.$ \newline where $(B_t)_{t \ge 0}$ is a standard Brownian Motion with respect to a filtration $({\cal{F}}_t)_{t \ge 0}$, $\mu, \sigma$ two continuous functions such that there exists $K_1 > 0$, $K_2 > 0$, such that $\forall t \ge 0$, $\lvert \mu(t) \rvert \le K_1$, $K_2 \le \lvert \sigma(t) \rvert \le K_1$.
\paragraph{Resolution - existence of a solution to the generalized Black-Scholes SDE}
Let us consider the following generic stochastic differential equation : \\
$\left\{
\begin{array}{l}
\mathrm{d}X_t = f(t,X_t) \mathrm{d}t + g(t,X_t) \mathrm{d}B_t\\
X_0 = x_0
\end{array}
\right.$ \\ where $(B_t)_{t \ge 0}$ is a standard Brownian Motion with respect to a filtration $({\cal{F}}_t)_{t \ge 0}$, $x_0 \in \mathbb{R}$, $f,g : \mathbb{R}_{+} \times \mathbb{R} \longrightarrow \mathbb{R}$ are jointly continuous in $(t,x)$ and Lipschitz in x. Then, by a theorem from Stochastic Calculus, we know that there exists a unique solution $(X_t)_{t \ge 0}$ to the SDE. in the case of the generalized Black-Scholes SDE, the conditions are met and we can thus conclude that it admits a unique solution. The solution can be made explicit here too, fortunately !
\paragraph{Resolution}
Let us set $Y_t = \log(S_t)$, we then have $\mathrm{d}Y_t = \frac{1}{S_t}\mathrm{d}S_t -\frac{1}{2} \frac{1}{S_t^2} \mathrm{d}<S>_t$. If we remember that $\mathrm{d}S_t = \mu(t) S_t \mathrm{d}t + \sigma(t) S_t \mathrm{d}B_t$ and apply the Isometry formula, we get that $\mathrm{d}<S>_t = \sigma(t)^2 X_t^2 \mathrm{d}t$. We thus get : \newline
\begin{equation}
\begin{alignat*}{2}
\sigma \mathrm{d}Y_t &= \frac{1}{S_t} \mathrm{d}S_t - \frac{1}{2} \frac{1}{S_t^2} \mathrm{d}<S>_t \\
&=  \frac{1}{S_t} \mathrm{d}S_t - \frac{1}{2} \sigma(t)^2 \mathrm{d}t \\
&=  \frac{1}{S_t} (\mu(t) S_t \mathrm{d}t + \sigma(t) S_t \mathrm{d}B_t)- \frac{1}{2} \sigma(t)^2 \mathrm{d}t \\
 &= (\mu(t) - \frac{1}{2} \sigma(t)^2) \mathrm{d}t + \sigma(t) dBt \\
\implies Y_t &= y_0 + \int_0^t \! (\mu(s) - \frac{1}{2} \sigma(s)^2) \, \mathrm{d}s + \int_0^t \! \sigma(s) \, \mathrm{d}B_s \\
\implies Y_t &= \log(s_0) + \int_0^t \! (\mu(s) - \frac{1}{2} \sigma(s)^2) \, \mathrm{d}s + \int_0^t \! \sigma(s) \, \mathrm{d}B_s \\
\implies S_t &= s_0 \exp(\int_0^t \! (\mu(s) - \frac{1}{2} \sigma(s)^2) \, \mathrm{d}s + \int_0^t \! \sigma(s) \, \mathrm{d}B_s)  \\
\end{alignat*}
\end{equation}\newline
Let us observe that the solution found in the case of the generalized Black-Scholes SDE is coherent with the solution found for the simple Black-Scholes SDE\footnote{Just set functions $\mu$ and $\sigma$ equal to constants $\mu$ and $\sigma$ and we are back with the Geometric Brownian Motion previously found. }.

\section{Back to the data}



