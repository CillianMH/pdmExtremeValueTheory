%\begingroup
%\let\cleardoublepage\clearpage


% English abstract
\cleardoublepage
\chapter*{Abstract}
%\markboth{Abstract}{Abstract}
\addcontentsline{toc}{chapter}{Abstract (English/Français)} % adds an entry to the table of contents
% put your text here
\bigskip
\paragraph{}
The topic of this project is twofold. It is both an introduction to extreme value theory and an application to finance.
\paragraph{}
First, we give an introduction to extreme value theory by making appear the extreme values distributions from the maxima of sequences of independent identically distributed random variables. This leads us to the two extremal problems and to the Fisher-Tippett-Gnedenko and Von Mises' theorems which provide answers to those problems.
\paragraph{}
Then, we study the theory of the Black-Scholes model for stock prices. We use a Black-Scholes model with time independent coefficients to simulate the prices over time of five stocks listed on the Paris stock exchange (BNP Paribas, Carrefour, LVMH, Sanofi and Total). Finally, we show that if we consider the stock prices above a high threshold (the $95$ \%-quantile for instance), we can fit extreme values distributions to the data.
\vskip0.5cm
%Key words: 
%put your text here


% French abstract
\begin{otherlanguage}{french}
\cleardoublepage
\chapter*{Résumé}
%\markboth{Résumé}{Résumé}
% put your text here
\bigskip
\paragraph{}
Notre sujet a une thématique duale, il est con\c cu comme une introduction à la théorie des valeurs extrêmes en statistique doublée d'une application au domaine de la finance.
\paragraph{}
Notre présentons d'abord l'approche en théorie des valeurs extrêmes, suivie d'une découverte des distributions des valeurs extrêmes à travers l'étude des maxima d'échantillons de variables aléatoires indépendantes identiquement distribuées. Cela nous amène aux théorèmes de Von Mises et de Fisher-Tippett-Gnedenko, ils constituent les réponses aux problèmes extrêmaux c'est-à-dire le coeur de la théorie des valeurs extrêmes.
\paragraph{}
Nous nous intéressons alors aux évolutions des prix de 5 actions listées à la bourse de Paris (BNP Paribas, Carrefour, LVMH, Sanofi et Total). Nous étudions d'abord la théorie du modèle de Black-Scholes, nous appliquons ensuite un modèle de Black-Scholes avec coefficients indépendants du temps à nos données. Enfin, nous réalisons la jonction entre les valeurs extrêmes et nos données financières en étudiant, pour chacune des 5 actions, les valeurs des prix dépassant un certain seuil (le $95$ \%-quantile, par exemple). Nous montrons qu'il est possible de procéder à un ajustement de courbes à l'aide de distributions généralisées des valeurs extrêmes sur ces données.
\vskip0.5cm
%Mots clefs: 
%put your text here
\end{otherlanguage}


%\endgroup			
%\vfill
