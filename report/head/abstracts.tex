%\begingroup
%\let\cleardoublepage\clearpage


% English abstract
\cleardoublepage
\chapter*{Abstract}
%\markboth{Abstract}{Abstract}
\addcontentsline{toc}{chapter}{Abstract (English/Français)} % adds an entry to the table of contents
% put your text here
\bigskip
\paragraph{}
The topic of this project is twofold. It is both an introduction to extreme value theory and an application to finance.
\paragraph{}
First, we give an introduction to extreme value theory by making appear the extreme values distributions from the maxima of sequences of independent identically distributed random variables. This leads us to the two extremal problems and to the Fisher-Tippett-Gnedenko and Von Mises' theorems which provide answers to those problems.
\paragraph{}
Then, we study the theory of the Black-Scholes model for stock prices. We use a Black-Scholes model with time independent coefficients to simulate the prices over time of five stocks listed on the Paris stock exchange (BNP Paribas, Carrefour, LVMH, Sanofi and Total). 
\paragraph{}
Finally, we consider the log-returns and the minus log-returns of those stocks. Taken above a high threshold (the $99$ \%-quantile, for instance), the log-returns turn out to fit extreme value distributions. The same is true for the minus log-returns taken below a low threshold. Last but not least, we show that in extreme value settings (values above high quantiles or below low quantiles), the Black-Scholes model is not valid.
\vskip0.5cm
%Key words: 
%put your text here


% French abstract
\begin{otherlanguage}{french}
\cleardoublepage
\chapter*{Résumé}
%\markboth{Résumé}{Résumé}
% put your text here
\bigskip
\paragraph{}
La thématique de notre projet est double : il est con\c cu comme à la fois une introduction à la théorie des valeurs extrêmes en statistique et une application au domaine de la finance.
\paragraph{}
Notre présentons d'abord l'approche en théorie des valeurs extrêmes, suivie d'une découverte des distributions des valeurs extrêmes à travers l'étude des maxima d'échantillons de variables aléatoires indépendantes identiquement distribuées. Cela nous amène aux théorèmes de Fisher-Tippett-Gnedenko et de Von Mises. Ils apportent les réponses aux problèmes extrêmaux qui constituent le coeur de la théorie des valeurs extrêmes.
\paragraph{}
Nous nous intéressons alors aux évolutions des prix de 5 actions cotées à la bourse de Paris (BNP Paribas, Carrefour, LVMH, Sanofi et Total). Nous étudions la théorie du modèle de Black-Scholes, et nous appliquons ensuite un modèle de Black-Scholes avec coefficients indépendants du temps à nos données. 
\paragraph{}
Enfin, nous considérons les log-retours et les moins log-retours de ces actions. Les log-retours au-dessus d'un seuil élevé ($99$ \%-quantile par exemple) se prêtent à un ajustement de courbes à l'aide de distributions généralisées des valeurs extrêmes. On observe un résultat analogue avec les moins log-retours au-dessous d'un seuil bas. Nous montrons alors que le modèle de Black-Scholes n'est pas satisfaisant en contexte d'extrêmes.
\vskip0.5cm
%Mots clefs: 
%put your text here
\end{otherlanguage}


%\endgroup			
%\vfill
