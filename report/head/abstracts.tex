%\begingroup
%\let\cleardoublepage\clearpage


% English abstract
\cleardoublepage
\chapter*{Abstract}
%\markboth{Abstract}{Abstract}
\addcontentsline{toc}{chapter}{Abstract (English/Français)} % adds an entry to the table of contents
% put your text here

\vskip0.5cm
%Key words: 
%put your text here


% French abstract
\begin{otherlanguage}{french}
\cleardoublepage
\chapter*{Résumé}
%\markboth{Résumé}{Résumé}
% put your text here
\bigskip
\paragraph{}
Notre sujet a une thématique duale, il est con\c c comme une introduction à la théorie des valeurs extrêmes en statistique doublée d'une application au domaine de la finance.
\paragraph{}
Notre présentons d'abord l'approche en théorie des valeurs extrêmes, suivie d'une découverte des distributions des valeurs extrêmes à travers l'étude des maxima d'échantillons de variables aléatoires indépendantes identiquement distribuées. Cela nous amène aux théorèmes de Von Mises et de Fisher-Tippett-Gnedenko, ils constituent les réponses aux problèmes extrêmaux c'est-à-dire le coeur de la théorie des valeurs extrêmes.
\paragraph{}
Nous nous intéressons alors aux évolutions des prix de 5 actions listées à la bourse de Paris (BNP Paribas, Carrefour, LVMH, Sanofi, Total). Nous étudions d'abord la théorie du modèle de Black-Scholes, que nous appliquons par la suite à nos données.
\paragraph{}
Enfin, nous réalisons la jonction entre les valeurs extrêmes et nos données financières en étudiant, pour chacune des 5 actions, les valeurs des prix dépassant un certain seuil. Nous essayons alors de procéder à un ajustement de courbe à l'aide de la distribution généralisée des valeurs extrêmes.
\vskip0.5cm
%Mots clefs: 
%put your text here
\end{otherlanguage}


%\endgroup			
%\vfill
